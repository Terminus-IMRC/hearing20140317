\documentclass[headrule,footrule,landscape,a4paper,25pt]{foils}

\title{整数問題プログラムの \\ メニーコアアーキテクチャにおける最適化}
\date{2014年3月17日}
\author{並木中等教育学校4年次 杉崎行優 \\ (チーム名:ARARAT)}

\begin{document}
\MyLogo{}
\maketitle

\foilhead{プロジェクトの概要}
 \MyLogo{整数問題プログラムのメニーコアアーキテクチャにおける最適化 \hfill ARARAT \hfill 並木中等教育学校4年次 杉崎行優}
 \begin{itemize}
  \item メニーコアアーキテクチャでの大規模並列処理と、木探索問題の並列処理を組み合わせた並列処理研究
  \begin{itemize}
   \item 複数ノードの数千コアでの並列処理における処理分散・通信の最適化
   \item Xeon Phi の性能は倍精度演算で表されており、整数演算の性能が理想的に出るかは不明
  \end{itemize}
 \end{itemize}

\foilhead{プロジェクトの計画(1)}
 \begin{itemize}
  \item 整数問題プログラムとして、平成25年度のT2K-Tsukuba学際利用で開発・並列化したn次魔方陣全解プログラムをベースに改良
  \begin{itemize}
   \item 並列方式はmaster-worker
   \item 通信の頻度・データサイズや各ワーカーの問題の粒度を比較的自由に変更可能
   \item 出力結果の成否判定が容易
  \end{itemize}
  \item 中間報告会までにはXeon Phi化し、それ以降に細かなチューニングや性能測定
 \end{itemize}

\foilhead{魔方陣について}
 \begin{itemize}
  \item 縦$n$横$n$のマスに$1$から$n^2$の数が$1$つずつ入れられており、
        縦の$n$列、横の$n$列、斜めの$2$列のそれぞれの数字の合計が全て等しく$L$(式\ref{eqn:l-of-ms})となるもの
 \end{itemize}
 \begin{equation} \label{eqn:l-of-ms}
  L=\frac{1}{n} \sum_{i=1}^{n^2}i = \frac{1}{2} n(n^2+1)
 \end{equation}

\foilhead{魔方陣の例($n=5$; $L=65$)}
 {\Large
  \begin{center}
   \begin{tabular}{|c|c|c|c|c|}
    \hline
    1 & 23 & 15 & 24 & 2 \\\hline
    19 & 3 & 17 & 4 & 22 \\\hline
    14 & 12 & 20 & 10 & 9 \\\hline
    13 & 21 & 8 & 16 & 7 \\\hline
    18 & 6 & 5 & 11 & 25 \\\hline
   \end{tabular}
  \end{center}
 }

\foilhead{プロジェクトの計画(2)}
 \begin{itemize}
  \item 開発 $\Rightarrow$ 実行のプロセスを数十回繰り返すと仮定
  \begin{itemize}
   \item 1回の並列実行は2時間程度であると予想
   \item 希望計算時間を600時間に変更
  \end{itemize}
  \item 他の整数問題への適用を目指す
 \end{itemize}

\end{document}
